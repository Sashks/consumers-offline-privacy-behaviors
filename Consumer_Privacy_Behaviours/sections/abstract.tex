\thispagestyle{empty}\section*{\ifthenelse{\boolean{english}}{Abstract}{Zusammenfassung}}

%------------------------------
As the use of technology becomes increasingly ubiquitous in our daily lives,  privacy concerns have become more prominent. This study aims to investigate the offline and online privacy behaviors of consumers to provide insights into how individuals protect their personal information in their day-to-day life. A qualitative research design was used, which involved conducting 39 semi-structured interviews, ensuring a heterogeneous sample, and 4 focus groups with participants knowledgeable in online technologies. The study identified a range of offline and online privacy behaviors, such as shredding documents, using strong passwords, and avoiding public Wi-Fi. The results also revealed that offline and online privacy behaviors are interconnected. The study's implications suggest that policymakers and businesses providers need to understand consumers' privacy behaviors to provide better protection for personal information. Future research can investigate the impact of contextual factors such as culture, age, gender, and personality on privacy behaviors and determine the effectiveness of different privacy behaviors in safeguarding personal information. Despite some limitations, such as small sample size and reliance on self-reported data, this research adds valuable insights to the literature on the privacy behaviors of consumers.