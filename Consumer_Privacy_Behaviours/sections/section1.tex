\section{Introduction}
\label{sec:1}
Privacy is now a major concern in politics, society, and for individuals. Surveys report that 67\% of internet users worldwide are more concerned about their online privacy than ever before \cite{one}. However, managing personal information online is becoming increasingly difficult, and as a result, 79\% of online users worldwide feel they have lost all control over their personal information \cite{one,two}. Repeated consumer data breaches have given people a sense of futility, ultimately making them weary of having to think about online privacy \cite{two}. Some internet users have given reasons for their privacy concerns: (1) Polls have found that most internet users are concerned about how websites handle their personal information \cite{three}; (2) Online privacy policies are often not easy to understand, legally confusing, containing a lot of jargon, and requiring strong reading skills \cite{four}.

Furthermore, Internet users have raised several points about these issues, arguing that: (1) 60\% of users who provide false personal information say they would be willing to provide their real information if the site told them how it would be used \cite{five}; (2) their privacy concerns could be reduced if websites provided privacy policies that were easy to understand \cite{three}. Despite the widespread assumption in the field of Privacy Enhancing Technologies (PETs) that users can manage and protect their privacy in a conscious way, the reality is that they don't know what their behaviors mean, and 29\% of users say they have never taken any action to protect their personal information \cite{six}. The contradiction between assumptions and reality results in PETs leading to a method of protecting privacy that does not work well. Therefore, we propose the following questions in our study:

\textit{Q1: What privacy behaviors are people engaging in offline, and what are the expectations behind them?}

\textit{Q2: What are the online equivalents of these privacy behaviors, and what are the expectations behind them?}

\textit{Q3: How can the results improve the performance of PETs?}

To answer these questions, we conducted two methods: interviews and focus groups. With our approach, we aim to gain more clarity on what privacy behaviors consumers are taking offline and online. These results can guide us in improving the performance of PETs to serve the privacy field and better protect the privacy of consumers.

The remainder of this paper is structured as follows: Section 2 provides a brief introduction to existing research on privacy, privacy behavior, and PETs. Section 3 gives an overview of our research methodology. Section 4 highlights the main results of our research approach. In section 5, we conduct a discussion to present the principle findings, implications, limitations, and future research directions. Finally, in section 6, we end with a brief conclusion.


% Beispiel für die Verwendung eines Abkürzungsverzeichnisses. Bitte entkommentieren:
%Das \gls{aifb} gehört zum \gls{kit}.

