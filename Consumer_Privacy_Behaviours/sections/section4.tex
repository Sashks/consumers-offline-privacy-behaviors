\section{Results}
\label{sec:4}

\subsection{Interview Results}
\label{Label2}

In this section, we report the findings of our study on consumers' offline privacy behaviors. Our sample consisted of 39 interviewees from diverse backgrounds and ages (see Table 4), with an average interview duration of 15-30 minutes. We identified more than 80 examples of offline privacy behaviors, each consisting of action, expectation and situation. Concretely, this resulted in 83 different actions, 21 expectations and 13 situations. Each behavior we categorized into one of three general groups: personal information, personal space or specific scenarios (see Tables 6-11). 

The interviewees reported a variety of actions they take to protect their offline privacy, ranging from physical actions to verbal interactions. The most reported actions were “pretending to be on the phone”, “pretending to not know the language”, and “pretending to be in a hurry”. These actions were often taken in situations where the interviewee wanted to prevent others from bothering them or taking their personal time. Other actions, such as wearing a mask or headphones, “taking more congested paths”, or “using stairs instead of elevators”, were reported to avoid attention or to protect personal space. 

The expectation behind these actions varied, but most interviewees reported aiming to protect their personal information, space, and resources. Interviewees also reported wanting to avoid feeling uncomfortable, ashamed, or unsafe. Some interviewees reported incentivizing or expecting reciprocity from others to protect their privacy. Additionally, interviewees reported taking actions to protect their purchase records, shopping habits, and ideas. 

The actions and expectations reported by interviewees were often situational. The most reported situations were on the street, in public places, on public transport, at work, at school, and at home. Other situations reported included shopping, withdrawing cash, paying, handling documents, travelling, interacting with technology, and communicating with people. Interviewees also reported taking actions to protect their privacy in dangerous situations, such as when staying in hotels or when lending their belongings. Some interviewees reported taking actions to protect their privacy while sharing or socializing with friends and family. 

In summary, our study identified more than 80 examples of offline privacy behaviors taken by consumers in a variety of situations. These behaviors were reported to protect personal information, personal space, and resources, and to avoid feeling uncomfortable, ashamed, or unsafe. While the actions taken by interviewees varied, most were situational and were taken in public places, at work or school, or at home. 

\subsection{Focus Group Results}
\label{Label2}

To gather information on the online equivalents of offline behaviors, we conducted focus groups with 11 participants divided into four homogeneous groups of 2-3 individuals each (see Table 5). The participants were all familiar with online technologies, which allowed for a more in-depth discussion of their online behaviors. 
During the 30–40-minute focus group sessions, we provided the participants with small sets of offline privacy behavior examples from the interview phase. We asked them to think of equivalent online behaviors that they are aware of or use to protect their digital (online) privacy. The participants provided around 65 examples, which can also be grouped into several categories based on their common themes (see Table 12). 

In terms of privacy-enhancing technologies, participants reported using a variety of tools to block unwanted notifications and cookies. They also used VPNs, ad blockers, and popup blockers to prevent unwanted tracking and ads. To protect their accounts, participants reported using strong passwords, two-factor authentication, and password managers. Additionally, they used secure browsers, anti-virus software, and browser containers to protect their online activities. 

Participants also reported avoiding certain online activities, such as not clicking on unfamiliar links, not accepting friend requests from strangers, and not posting their location or vacation plans online. They also mentioned the importance of using secure networks and not using public computers for sensitive activities. 

Overall, the results of our focus groups suggest that there exists a wide range of online behaviors to protect personal information and resources. These behaviors range from simple actions like blocking notifications or not clicking on unfamiliar links, to more complex measures like using two-factor authentication and private search engines. As with offline behaviors, people's online privacy and security behaviors are shaped by their level of risk awareness and their perceived threats in the digital environment. 

\subsection{Offline behaviors and Online Equivalents}
\label{Label2}

There are some offline behaviors that can be directly mapped to their online equivalents, while others may not have a direct correlation. For example, behaviors such as "pretending to not know the language" and "putting the spotlight on someone else" may not have a direct online equivalent, while "not sharing personal information" and "avoiding weak PINs" have clear online counterparts. 

Additionally, some offline behaviors can have multiple online equivalents depending on the context. For instance, "pretending to be in a hurry" was mapped by participants to "using do not disturb mode", "using airplane mode" or "replying later". In other words, an online behavior like using a VPN can be seen as equivalent to all offline behaviors, whose expectation involves “protecting information”. Consequently, the online examples are to be considered as equivalents more regarding the common expectation of multiple concrete offline actions as opposed to a "one on one" mapping between online and offline privacy behaviors. 

Overall, while there is an overlap between offline and online privacy behaviors, they are not always directly comparable, and it is important to consider the context in which these behaviors occur. 