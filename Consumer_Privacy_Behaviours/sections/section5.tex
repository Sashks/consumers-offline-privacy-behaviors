\section{Discussion}
\label{sec:5}

\subsection{Principal Findings}
\label{Label2}


\subsubsection{Phase 1 - Interviews} 
After investigating first phase interview, we categorize the main private objectives of our interviewees which are personal space, personal identity and information and specific scenarios. Next step, we analyze the relative private behaviors and categorize them to the belonged categories. The findings of first phase interview as follows:  

Personal Space:

In this category, generally, we analyze our interviewees that they have the private need to keep their personal space over certain distances. How do they behave so that they can keep their distance from others? They will create boundaries to protect their personal space. Furthermore, we explain the definitions of distance into two terms with more details - physical and mental distance in order to interpret the results more precisely. According to our respondents, they create physical distances to hold their personal zone, for example, work from home and avoid rush hour to decrease connections with others. The other example is to put their personal belongings in public transport to create distances and make them  more comfortable to stay in their personal seat. On the other hand, people wear masks and headphones to set up invisible boundaries to avoid social contacts. Besides, switching languages is also part of protecting behaviors to escape attentions of their real intentions. Besides, they view their personal room as their personal zone, therefore, locking the door and closing windows are their protections for their privacy. 

Personal Identity and Information:

In the second category, personal identity and information are concerned as personal extensions from our interviewees. Thus, to assure their positive personal identity and information is one of their great considerations. At the same time, they tend to evade any risks to damage their personal image. Other behaviors are also mentioned that using watermark is a safeguard for personal intellectual property. In the other case, the interlocutors type their judgements via messages instead of conversations when they discuss confidential topics. The meaning behind this behavior is they do not wish any exposure of their intent to others. 

Specific scenarios: 

We summarize the essential samples to specific scenarios which are indicated significantly from our interviewees. Firstly, monetary repercussions have most of concerns from our interviewees. Hiding personal monetary information is the top mission to keep their personal momentous asset from danger. They cover their hand when they type their PIN numbers. In addition, we discover that purchase record is deemed as personal identity as well. As a result, some respondents do not apply for any membership card in order to unveil personal purchasing data, for instance. The reason is they expect service providers collect their personal purchase data for marketing use, so they doubt their purchase data in safe position. The second scenarios is avoiding ads. The most of interlocutors lean to ignore noises and interruptions, consequently they neglect the notifications from the Ads which they do not interest. Thirdly, taboo topics are seen as embarrassing themes which are relative to sexual orientation, personal diseases and real emotions. People tend to stop sharing and change to other acceptable subjects to evade the embarrassed feeling. 

\subsubsection{Phase 2 - Focus Groups}
Before we started to interpret the results from second phase focus group, there are additional knowledge of Current Privacy Protection: Privacy-enhancing Technologies (PETs) must be acknowledged in order to map the second phase result .Privacy-enhancing Technologies (PETs) belong to a class of technical measures which aim at preserving the privacy of individuals or groups of individuals. Their goal is to protect user identities by providing anonymity, pseudonymity, unlinkability, and unobservability of users as well as data subjects \cite{twelve}. Some of the goals they have are \cite{priv_goals}:

Privacy Goal 1: Data Minimization 

Privacy Goal 2: Data Anonymization 

Privacy Goal 3: Transparency, Consent, and Verifiability 

Additionally, the current aspect of information privacy also guides us to perceive the gaps between users and service providers. The current aspect of information privacy is as a social contract in 2016  due to diversity of users’ information privacy needs 

\begin{table}[H]
    \centering
    \begin{tabular}{ |p{1cm}|p{7cm}|p{7cm}|  }
\hline
    Year & Perspective & Cue \\
\hline
\hline
     1890 & Information privacy as a right \cite{seven} & Increasing prevalence of newspapers\\
     & & \\
\hline
     1967 & Information privacy as control \cite{table2} & Computerization of government databases\\
\hline
    1996 & Information privacy as commodity \cite{priv_goals} & Commercialization of information networks\\
\hline
    2002 & Information privacy as a set of related problems \cite{table4} & Absence of a universal conceptualization of information\\
\hline
    \textbf{2016} & \textbf{Information privacy as social contract} \cite{table5} & \textbf{Diversity of users' information privacy needs}\\
\hline
    \end{tabular}
    \caption{History of information privacy \cite{cii}}
    \label{tab:meng_t1}
\end{table}

We restate three same categories from first interview to map and frame the second phase interview results as follows: 

(1) Personal Space:

Respondents use private tools, private router & mode and VPN to establish private zone. The purpose is to construct a safe place in the digital world and avoid attention and interruption. To cite an instance, people tend to set up offline state (ex: MS team) even when they are online. There is a specific finding that people mentioned the reactions from social posts are more aggressive, so they attend to take extra care of what they posted, and they block the comment sections from the social media in order to evade humiliation. 

(2) Personal Identity:

Interlocutors tend to fill only necessary and appreciate personal data which matches only the purpose of collection. The purpose is to prevent disclosure of unnecessary personal information. In the other case, people pretend even to create a fake account or fake personal information (ex: name, birthday...), because they do not trust the purpose of collection and their personal data will be kept in safety. The rest of the behaviors are to erase browser history, use code words instead key words to decrease attentions from others. 

(3) Specific scenarios:

In monetary repercussions, the interesting finding is that people behave differently for the same purpose to maintain monetary safety. Some people are more willing to pay online because they trust the online payment system. However, some others are more likely to pay in cash because they consider cash is reducing the risks of leaking personal monetary information. The second scenario is avoiding ads. Respondents set up online blockers (ex: ads, Cookie and Pop-up blockers) to reduce the disturbances. The third scenario is taboo topics which are relative to sex orientation, personal diseases, real emotions. People use private mode, VPN to care for viewing history. In other respects, some people use the code words instead of key words during conversation or messages, for instance, "happy water" replace with "alcohol". 


\subsubsection{Mapping Online to Offline behaviors}

\begin{table}[H]
    \centering
    \begin{tabular}{ |p{2cm}|p{2.8cm}|p{3.2cm}|p{3.5cm}|p{3cm}|  }
\hline
    Category & Expectations & Offline behaviors & Online behaviors & Finding \\
\hline
\hline
     Personal Space & Offline: Create boundaries & Physical: WFH & Private Mode, VPN & \textbf{Invisible online world*} \\ \cline{2-5}
     & Online: Avoid interruption and humiliation & Mental: mask, headphones & No personal status (MS) & \textbf{Insecurity} \\\cline{4-5}
     & & & Block comments & \textbf{Less social manners} \\
\hline
     Personal Identity and Information & Protect personal thinking and identity & Texting instead of talking, Watermark & Erase browser history, Use fake account & \textbf{Lower ethic awareness of real self-identity in online world*}\\
\hline
    Monetary repercussions & Hide confidential information & No purchase record, cover pin & Use secure network, \textbf{Pay online vs. Pay in cash} & Good privacy protection in both\\
\hline
    Advertise- ments(ads) & Avoid interruption & Ignore street ads & Use ad blockers & Systemic online protections \\ \cline{5-5}
    & & & & \textbf{Accept/reject cookies has the same expectation}\\
\hline
    Taboo topics & Avoid embarrassment & No sharing & Use private mode, \textbf{code words} & Good privacy protection in both\\
\hline
    \end{tabular}
    \caption{Mapping Online to Offline behaviors}
    \label{tab:meng_t1}
\end{table}

After investigating our first and second phase of interviews. If there are any equivalents between offline and online privacy behaviors? The answer is yes! There are some interesting findings from comparisons of both interviews. From the categories of personal space and personal identity and information, we detected that users have less trust, security and social manners and lower ethical awareness of real self-identification in the online world. It may be due to the isolation and lower awareness of virtual world, especially when users tend to do things in private mode and lock the door and window at the same time, people behave more real and have less concerns without social mask. 

Moreover, we conclude that users' expectations for offline and online private preferences, we indicate that users have same needs in both areas as follows: 

(1) Safety in personal space (ex: no interruption) 

(2) Protect personal identity and information 

(3) Protect personal concerns for specific scenarios 

(4) No risk in leaking and hacking personal information 

On the other hand, the interviewees response specifically that privacy protections are not always working. Due this finding we conclude the reasons as follows: 

Users do not have enough privacy knowledge to judge the privacy protections  

Users have doubts that service providers can manipulate protections 

Hacking news lowers the users trust of IT security.  

Every user has personal privacy priorities, so private preferences are different. 

Afterwards, we map the results from second phase interviews and PETs to evaluate if there is any gap between the users’ private expectations and the policy. Furthermore, we propose potential solutions for improving the gap of expectations from users and PETs as well. 

 
\subsubsection{Mapping Results to PETs}

\begin{table}[H]
    \centering
    \begin{tabular}{ |p{3cm}|p{4cm}|p{2cm}|p{5cm}|  }
\hline
    PET Goals & Users' Expectation & Matching & Reason/Concern \\
\hline
\hline
    Data Minimization & Protect personal identity and information & \checkmark & Users can make their own decisions well \\
\hline
    Data Anonymization & No personal identity and information & \checkmark & What does private mode really protect?\\
\hline
    \textbf{Transparency, Consent, Verifiability} & \textbf{Easy to understand} & \text{\sffamily X} & \textbf{Do not have enough knowledge to understand users' term} \\\cline{4-4}
    & & & \textbf{Trust issues: manipulation} \\ \cline{4-4}
    & & & \textbf{IT-security issues: news from hacking} \\ 
\hline
    \textbf{Confidentiality of the goal (see suggestions)} & \textbf{Data Confidentiality} & \text{\sffamily ?} & \textbf{No leaking and hacking} \\ 
\hline
    \textbf{New Goal (see suggestions)} & \textbf{Protect personal preferences in certain topics} & \text{\sffamily ?} & \textbf{Protect personal concerns for specific scenarios} \\ \cline{4-4} 
    & & & \textbf{Personal privacy preferences are different} \\
\hline
     
    \end{tabular}
    \caption{Mapping Results to PETs}
    \label{tab:meng_t1}
\end{table}

Data minimization: PETs are matching users’ expectations which users believe they can control the willing to give personal data for certain purposes. 

Data anonymization: PETs are matching mostly of users’ expectations. The most of users use private mode as their private protection. However, when we ask more deeper of this question which is how much do they know about the private mode or what does exactly protect in private mode. All the answers are no. 

Transparency, consent and verifiability: The most of users mentioned that they can not understand the users’ terms. Secondly, users do not trust  generally service providers which they believe service providers can manipulate the policy and make it hard to understand. At the same time, current hacking news lower their trust of IT security to the service providers. 

Extensional goals: The additional suggestions are considered the gap of users’ expectations and PETs which indicates users’ needs and expectations are not satisfied yet that can be additional improvement to current policy. 

Confidentiality: Users still have strong uncertainty of data leaking and hacking risks. As a result, the solutions of integration to the safer protections are urgent to improve and develop. 

Personal preferences: Due the diverse concerns within topics, everyone has individual  tolerances and opinions regarding various subjects. Therefore, it is not appropriate to set the policy for everyone in the same page, so we propose to add personal preferences in the policy. 

\subsection{Implications}
\label{Label2}

After analyzing the results from both interviews, we have some suggestions that could help improve PETs’ performance. 

For internet users:  

We recommend that users do not overestimate capabilities when they have any problems. Searching for professional help will be more practical for solutions which are provided by experts. 

In the other side, to empower necessary knowledge is a powerful solution. Therefore, we recommend enhancing users to be capable and competent actors in digital work, for instance, online privacy education in order to improve their skills to solve the problems. 

For service provider: 

Make the users' policy, terms and agreements are easy to understand. 

Open the channels to interact with users for feedback, help and emergency rescues. When users have any urgent problems, they know where to search for support from service providers. 

For government: 

Set up more online polices for avoiding online criminal, privacy and IT security Direct the way when users have problem and attacks to online polices. Police in the real world represent justice, more online police bring more trust from users and fulfill their IT security expectations. 

\subsection{Future Research}
\label{Label2}

While this study has provided a valuable insight into the offline and online privacy behaviors of consumers, there is still much to be explored in this area. Future research could focus on: 

Exploring the relationship between offline and online privacy behaviors: While this study has identified some offline privacy behaviors and their online equivalents, there is still a need to investigate how consumers' offline privacy behaviors influence their online privacy behaviors and vice versa. 

Investigating the impact of contextual factors on privacy behaviors: This study has identified various situations in which consumers engage in privacy behaviors, but it has not explored the contextual factors that may influence these behaviors. Future research could investigate how factors such as culture, age, gender, and personality impact privacy behaviors. 

Examining the effectiveness of privacy behaviors: While this study has identified various privacy behaviors that consumers engage in, it has only subjectively assessed the effectiveness of these behaviors in protecting consumers' privacy. Future research could explore more in-depth the effectiveness of different privacy behaviors in protecting consumers' privacy in various contexts. 

Examining the impact of privacy regulations on privacy behaviors: This study has not explored the impact of privacy regulations on consumers' privacy behaviors. Future research could investigate how privacy regulations, such as the General Data Protection Regulation (GDPR \cite{gdpr}) and the California Consumer Privacy Act (CCPA \cite{ccpa}), influence consumers' privacy behaviors. 

Overall, there is still much to be learned about consumers' offline and online privacy behaviors, and future research in this area could help to inform policymakers, businesses, and consumers about how to protect privacy in an increasingly digital world. 

\subsection{Limitations}
\label{Label2}

Although our study provides a comprehensive list of offline and online privacy behaviors, it is important to acknowledge some limitations. First, our sample size for the interviewees was limited to 39 individuals, which may not be representative of the broader population. Additionally, while efforts were made to ensure a diverse sample, it is possible that some groups may be underrepresented in our sample. 

Second, the study relied on self-reported data from participants, which could be subject to bias or social desirability effects. Furthermore, the focus groups were composed of only 11 participants, which may limit the generalizability of our findings. 

Third, our focus groups included only individuals who are familiar with online technologies. Thus, our results may not generalize to individuals who are less tech-savvy or have limited access to online resources. 

Additionally, behaviors were grouped based on a general description of the participants' expectations. Future work can involve separation into more detailed categories based on the actions themselves, their purpose, meaning, impact, effectiveness or the situation in which they are performed. 

Finally, while our study identified a range of offline and online privacy behaviors, we did not examine in-depth the effectiveness of these behaviors in protecting individuals' privacy. Further research is needed to determine which privacy behaviors are most effective and how they can be improved. 

Despite these limitations, we believe that our study makes a valuable contribution to the literature on consumers' offline and online privacy behaviors and provides a solid foundation for future research in this area. 


