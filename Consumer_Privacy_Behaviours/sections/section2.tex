\section{Background}
\label{sec:2}

\subsection{Privacy and privacy protection}
\label{Label2} 
 

Warren and Brandeis defined privacy as “the right to be let alone” \cite{seven}. According to Ross, privacy is the ability and/or right to protect our personal secrets, the ability and/or right to prevent invading our personal space \cite{eight}. Privacy can be understood as a quasi “aura” around the individual, which constitutes the limit between him/her and the outside world \cite{nine}. 

Fischer-Hübner proposed four ways to protect privacy: protection by government laws; protection by privacy-enhancing technologies (PETs); self-regulation for fair information practise by codes of conduct promoted by businesses; privacy education of consumers and IT professionals \cite{ten}. 

In our research, we will further identify what privacy behaviors consumers will engage in to protect their behavior. At the same time, we will focus on how PETs can be improved to protect privacy. 

\subsection{Privacy Enhancing Technologies (PETs)}
\label{Label2} 

PETs are a system of ICT measures protecting informational privacy by eliminating or minimizing personal data thereby preventing unnecessary or unwanted processing of personal data, without the loss of the functionality of the information system \cite{eleven}. After the pseudo-identity was established, newer PETs gave rise to a classification in seven principles: Limitation in the collection of personal data, Identification/authentication/authorisation, Standard techniques used for privacy protection, Pseudo-identity, Encryption, Biometrics and Audit ability \cite{eleven}. PETs belong to a class of technical measures which aim at preserving the privacy of individuals or groups of individuals \cite{twelve}. PETs can help individual users control the amount of personal information they disclose in an online transaction \cite{thirteen}. The goal of PETs is to restore the balance of power between the individual who wants to retain privacy and many actors in the online environment who want to gather personal information \cite{thirteen}. Heurix also proposed the goal of PETs is to protect user identities by providing anonymity, pseudonymity, unlinkability, and unobservability of users as well as data subjects \cite{twelve}. 

A prerequisite for the development of PETs is to understand the privacy needs of consumers. Only a true clarity of consumer requirements and expectations for privacy can improve the performance of PETs. 
