\section{Methodology}
\label{sec:3}

\subsection{Interview}
\label{Label2}

Qualitative interviewing is a flexible and powerful tool to capture the voices and the ways people make meaning of their experiences \cite{fourteen}. 

In this phase, we hoped to gain insight into the private behaviors that consumers would engage in offline. 

In the first step, we first clarified the purpose and format of the interviews. The purpose of the interview is to explore offline privacy behaviors and find the expectations of these behaviors. We conducted semi-structured one-to-one interviews, i.e. after initially developing an interview guideline, the interview questions were adjusted accordingly as the interview progressed. The interview was initially assumed to last between 15-30 minutes. In the second step, we developed an interview guideline (see Appendix 1) and identified the interviewees, i.e. a heterogeneous sample (i.e. different ages, different occupations, different nationalities, different levels of education, etc.). In the third step, we conducted the interviews and recorded them through notes for subsequent analysis. At the same time, the specific questions of the interviews were gradually adapted as the interviews progressed. In the fourth step, we summarised and analysed the interview data to classify privacy behaviors, situations and expectations to provide the basis for the next stage of the research methodology.

\subsection{Focus Group}
\label{Label2}

In the second phase, we conducted four focus groups. We wanted to map the offline privacy behaviors obtained in the first phase to online privacy behaviors through focus groups, and to clarify the behavioral hidden expectations. 

In the first step, we selected participants who had more knowledge about privacy and technology as the target group for this phase. In the second step, we brainstormed to note the online equivalents of the offline behaviors as well as their expectations. In the third step, the results were summarised and the online privacy behaviors were categorised. 
